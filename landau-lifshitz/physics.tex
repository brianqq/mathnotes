\documentclass{scrbook}

\usepackage[dropmargin]{brian}
\usepackage{bm}
\usepackage[inline]{asymptote}
%\usepackage{chngcntr}
\newcommand{\hyper}[1]{\prescript{*}{}{#1}}
\newcommand{\standard}[1]{\prescript{\sigma}{}{#1}}
\DeclareMathOperator{\st}{st}
\DeclareMathOperator{\sn}{sn}
\newcommand{\infeq}{\simeq}

%no skip page after chapter
\usepackage{etoolbox}
\usepackage{etoolbox}
\makeatletter
\patchcmd{\chapter}{\if@openright\cleardoublepage\else\clearpage\fi}{}{}{}
\makeatother

\numberwithin{thms}{chapter}

\newcommand{\der}{\dot}
\newcommand{\dder}{\ddot}
\renewcommand{\vec}[1]{\bm{#1}}
\newcommand{\Der}{D}
 


\begin{document}
\chapter{Equations of motion}
\section{Lagrangian}
  \newcommand{\q}{\varq}
  \newcommand{\derq}{\der \q}
  \newcommand{\dt}{d\vart}
\begin{defn}[action]
  Let $\varL(\q,\derq, \vart)$ be a Lagrangian.
  Let $\q:\textrm{time}\to\textrm{configuration space}$. 
  Define action by 
  \[
    \varS(\q):=\int_a^b \varL(\q,\derq, \vart) \dt 
  \]
\end{defn}

Mechanics will minimize action of the appropriate Lagrangian.
\begin{theorem}[euler-lagrange]
  \label{th:euler-lagrange}
  \[
    \frac{d\varS}{d\q} = \inner{\frac{\partial \varL}{\partial \q} - \frac{d}{d\vart}\frac{\partial \varL}{d\derq}}{-}
  \]
\end{theorem}
\begin{proof}
  \newcommand{\dq}{h}
  \newcommand{\derdq}{h'}
  Let $\dq$ be an infinitesimal perturbation (endpoints are $0$). 
  \begin{align*}
    \varS(\q+\dq) &= \int \varL(\q+\dq,\derq + d\derq,t)\dt \\
           &= \int \varL(\q,\derq,\vart)\dt + \int \varD\varL(\varq,\derq, \vart)(\dq,d\derq,0) + o(\dq) \\
           &= \int \frac{\partial \varL}{\partial \q} \dq\dt + \underbrace{\frac{\partial L}{\partial \derq}}_{u} \underbrace{d\derq \dt}_{dv} \\
           &=\int\left(\frac{\partial \varL}{\partial \q} - \frac{d}{d\vart} \frac{\partial \varL}{\partial \derq}\right)\dq \dt + \underbrace{\left[\dq\frac{\partial L}{\partial \derq} \right]_a^b}_{\mathclap{0\textrm{ as $\dq(a)=\dq(b)=0$}}}
  \end{align*}
\end{proof}
\begin{cor}[stationary action]
  \[
    \frac{\partial \varL}{\partial \varq} - \frac{d}{d\vart}\frac{\partial \varL}{d\der \varq} = 0
  \]
\end{cor}

\chapter{Conservation laws}
\section{Energy}
\begin{defn}[energy]
  \label{def:energy}
  \[
    \varE\define  \inner{\Der \varq}{\frac{\partial\varL}{\partial \derq}} - L
  \]
\end{defn}
\begin{theorem}
  When $\Der(t\mapsto L(\_,\_,t)) =0$ (\eg closed system or constant field), energy is conserved. 
\end{theorem}
\begin{proof}
  By hypothesis, 
  \[
    \frac{d\varL}{d\vart} = \inner{\frac{\partial L}{\partial \varq}}{\der \varq} + \inner{\frac{\partial L}{\partial \derq}}{\dder \varq}
  \]
  By \nameref{th:euler-lagrange},
  \[
    \frac{d\varL}{d\vart} = \inner{\frac{d}{d\vart}\frac{\partial L}{\partial \der\varq}}{\der \varq} + \inner{\frac{\partial L}{\partial \derq}}{\dder \varq}
  \]
  By product rule, 
  \[
    \frac{d\varL}{d\vart} = \frac{d}{dt}\inner{\derq}{\frac{\partial L}{\partial \der \varq}}
  \]
  Subtracting $d\varL/d\vart$ finishes the proof. 
\end{proof}
\section{Momentum}
\begin{defn}[momentum]\label{def:momentum}
  \renewcommand{\varv}{\vec v}
  \renewcommand{\varr}{\vec r}
  \[
    P\define \sum_a \frac{\partial \varL_a}{\partial\varv_a}
  \]
  where $a$ indexes over particles. 
\end{defn}
\begin{theorem}[conservation of momentum]
  In a closed system, $\der P=0$. 
\end{theorem}
\begin{proof}
  \newcommand{\nudge}{\vec\varepsilon}
  \newcommand{\dL}{dL}
  \renewcommand{\varr}{\vec r}
  This follows from homogeneity of space. Suppose every particle moves by infinitesimal $\nudge$ with velocities unchanged. Then 
  \[
    \dL = \sum_a \inner{\frac{\partial \varL}{\partial L}{\partial \varr_a}}{\nudge}
  \]
  which must be $0$ by homogeneity. As $\nudge$ is arbitrary, we conclude 
  \[
    \sum_a \frac{\partial \varL}{\partial \varr_a}=0
  \]
  By \nameref{th:euler-lagrange},
  \[
    \frac{d}{d\vart}\sum_a \frac{\partial \varL}{\partial \varv_a} = 0
  \]
\end{proof}
\begin{cor}[Newton's third law]
  \[
    \sum_a \vec \varF_a =0 
  \]
\end{cor}
\begin{defn}[generalized momenta]
  \[
    \varp_i \define \frac{\partial \varL}{\partial \der\varq_i}
  \]
\end{defn}
\begin{defn}[generalized force]
  \[
    \varF_i \define \der \varp_i
  \]
\end{defn}
\section{center of mass}
\renewcommand{\varR}{\vec R}
\renewcommand{\varr}{\vec r}
\begin{defn}[Center of mass]
  \[
    \varR \define \frac{\sum \varm_a\varr_a}{\sum \varm_a}
  \]
\end{defn}
A \emph{system} is \emph{at rest} ($P=0$) in a frame of reference where $R=0$. 
\vspace{1ex}
\begin{defn}[internal energy]
  The energy of a system at rest, denoted $E_i$.
\end{defn}
\renewcommand{\varv}{\vec v}
\renewcommand{\varV}{\vec V}

\begin{theorem}
  Energy in different reference frames.
  Put $\varv_a'=\varv_a+\varV$. Then
  \[
    E=E'+\inner \varV {P'} + \frac12 \mu \varV^2
  \]
\end{theorem}
\begin{proof}
  \begin{align*}
    \varE&= \frac12 \sum m_a |\varv_a|^2 + U \\
         &=\frac12 \sum m_a |\varv_a'+\varV|^2 + U \\
         &=\frac12 \mu \varV^2  + \inner{\varV}{\sum m_a\varv_a'}+\frac 12 \sum m_a|\varv_a'|^2 +U
  \end{align*}
\end{proof}
\renewcommand{\varV}{|\vec V|}
\begin{cor}
  \[
    \varE = \frac{1}{2} \mu \varV^2+\varE_i
  \]
\end{cor}
\begin{cor}
  \[
    L=L'+\inner{\varv}{\varP'}+\frac 12 \mu \varV^2
  \]
\end{cor}
\begin{cor}
  \[
    S=S'+\mu\inner{\varv}{\varP'}+\frac12 \mu \varV^2t
  \]
\end{cor}
\section{Angular momentum}
\begin{theorem}
  $\so(3)$ is a Lie group with Lie algebra 
  \[
    \fso(3)\cong (\R^3)^*\wedge\R^3
  \]
\end{theorem}
\begin{proof}
  \newcommand{\tp}[1]{{#1}^T}
  Let
  \[
    \varf(\varx)\define \tp{\varx}\varx
  \]
  Then
  \[
    \so(3)\cong \varf^{-1}(I)
  \]
  But 
  \begin{align*}
    f(I+dx)&=\tp{(I+dx)}(I+dx) \\
           &=\tp I I + \tp {dx} I + \tp I dx + \tp {dx} dx 
  \end{align*}
  Hence
  \begin{align*}
    Df|_I(h)&= h + \tp h  \\
    Df|_x &= (x\_)Df|_I(x^{-1}\_)
  \end{align*}
  so
  \[
    \fso(3)\cong\ker Df \cong \R^3\wedge\R^3
  \]
\end{proof}
\renewcommand{\varr}{\vec r}
\renewcommand{\varp}{\vec p}
\renewcommand{\varP}{\vec P}
\renewcommand{\varf}{\vec {\dot p}}
\renewcommand{\varM}{\vec M}
\begin{defn}[angular momentum]
  \[
  \varM \define \sum \varr_a\times \varp_a
  \]
\end{defn}
\begin{theorem}[conservation of angular momentum]\label{th:cons-ang-mom}
  In a closed system,
  \[
    \der\varM=0
  \]
\end{theorem}
\begin{proof}
  Let $d\phi\in\fso(3)$ be an infinitesimal rotation.  
  \[
    d\varr=d\phi\times\varr
  \]
  But the Lagrangian is unchanged by $d\phi$, hence
  \begin{align*}
    dL=\sum_a \left(\frac{\partial L}{\partial \varr_a} d\varr_a +\frac{\partial L}{\partial \varv_a}d\varv_a\right) &= 0 \\
    \sum \inner{\varf_a}{d\phi\times\varr_a} + \inner{\varp_a}{d\phi\times\varv_a}&=0 \\
    \inner{d\phi}{\sum (\varr_a\times \varf_a+\varv_a\times \varp_a)}&=0\\
    \inner{d\phi}{\frac d{dt}\sum \varr_a\times \varp_a}&=0
  \end{align*}
  But $d\phi$ is arbitrary, hence
  \[
    \frac{d}{dt} \sum \varr_a\times \varp_a = 0
  \]
\end{proof}
\renewcommand{\vara}{\vec a}
\begin{theorem}[$M$ in a different frame]
  If $r_a=r_a'+\vara$, then
  \[
    \varM=\varM'+a\times \varP
  \]
\end{theorem}
\begin{proof}
  \begin{align*}
    \varM &= \sum \varr_a\times \varp_a \\
          &=\sum \varr_a'\times \varp_a + \vara\times\sum\varp_a \\
          &=\varM'+\vara\times\varP
  \end{align*}
\end{proof}
\renewcommand{\vare}{\vec e}
\begin{cor}
  If $\varL$ (equivalently, $\varU$) is symmetric around the axis with unit $\vare$, then $\inner{\varM}{\vare}$ is conserved. 
\end{cor}
\begin{cor}
  If $\varL$ is symmetric around the origin, then $\varM$ is conserved. 
\end{cor}
\chapter{Integration of the equations of motion}
\section{One dimension}
\[
  L=\frac 12 m\der x - U(x)
\]
so 
\[
  t=\int_{x_0}^{x_1} dx\sqrt{\frac{m/2}{E-U(x)}}
\]
\section{Pendulum}
\newcommand{\ang}{\phi}
\newcommand{\scl}{\omega^2}
Consider the pendulum
\[
  z = -ir \exp(i\ang)
\]
with
\begin{align*}
  K &= \frac12 m|\der z|^2 = \frac12 mr^2|\der\ang|^2 \\
  U &= -mgr\cos \ang  \\
  L &= r|\der\ang|^2 +g\cos\ang
\end{align*}
Let $\scl=g/r$.
By \nameref{th:euler-lagrange}
\begin{align*}
  \dder\ang + \scl\sin\ang &=0\\
  \der\ang\dder\ang +\der\ang \scl \sin\ang &= 0 \\
  \frac d{dt} \left( \frac{\der\ang^2}2-\scl\cos\ang\right)&= 0 
\end{align*}
Integrating gives 
\begin{align*}
  0&=\left( \frac{\der\ang^2}2-\scl\cos\ang\right)_0^t \\ 
  \der\ang^2 &=\left.2\scl\cos\ang\right|_0^t+\der\ang(0)^2
\end{align*}
Hence 
\[
  \pm t=\int_{\ang(0)}^{\ang} \frac{d\ang}{\sqrt{\left.2\omega^2\cos\ang\right|_0^t+\der\ang(0)^2}}
\]
But $\cos\phi|_0^t = 2\sin^2(\phi/2)|_t^0$, so 
\begin{align*}
  \pm t&=\int_{\ang(0)}^{\ang} \frac{d\ang}{\sqrt{\left.4\omega^2\sin^2\frac\ang2\right|_t^0+\der\ang(0)^2}} \\
  \pm 2\omega t&=\int_{\ang(0)}^{\ang} \frac{d\ang}{\sqrt{\sin^2\frac{\ang(0)}2+\frac{\der\ang(0)^2}{4\omega^2}-\sin^2\frac{\ang}2}} \\
\end{align*}
Let
\[
  k=\left(\sin^2\frac{\phi(0)}2 + \frac{\der\phi(0)^2}{4\omega^2}\right)^{-1/2}
\]
Then
\begin{align*}
  \pm \frac{2\omega t}{k} &= \int_{\phi(0)}^\phi \frac{d\ang}{\sqrt{1-k^2\sin^2\frac{\ang}2}}  \\
  \pm \frac{\omega t}{k} &= \int_{\phi(0)/2}^{\phi/2} \frac{d\theta}{\sqrt{1-k^2\sin^2\theta}}  \\
\end{align*}
\begin{defn}[Incomplete elliptic: 1st kind]
  \[
    F(u,k)=\int_0^u \frac{d\theta}{1-k^2\sin^2\theta}
  \]
\end{defn}
\begin{defn}[$\sn$]
  \[
    \sn(-,k)=\sin\left( F^{-1}(-,k)\right)
  \]
\end{defn}
So
\[
  \phi(t)=2\arcsin\sn\left(F\left(2^{-1}\phi(0),k\right)\pm \frac{\omega t}k,k\right)
\]
\section{Two body problem}
As the system is closed, the center of mass is inertial. Set this as the origin.
\[
  \frac{m_1\varz_1+m_2\varz_2}{m_1+m_2} = 0
\]
We would like to switch to the reference frame where the Sun is stationary:
\[
  \varz=\varz_2-\varz_1
\]
Solving in terms of $\varz$ gives 
\begin{align*}
  z_1&=\frac{m_2z}{m_1+m_2} \\
  z_2&=\frac{m_1z}{m_2+m_1} 
\end{align*}
To find $K$,
\begin{align*}
  K&=\frac12 m_1|\der z_1|^2+\frac12 m_2|\der z_2|^2 \\
   &=\frac12 m_1\left|\frac{m_2\der z}{m_1m_2}\right|^2+\frac12 m_2\left|\frac{m_1\der z}{m_2+m_1}\right|^2 \\
   &=\frac12\frac{m_1m_2(m_1+m_2)}{(m_1+m_2)^2}|\der z|^2 \\
   &=\frac 12 \underbrace{\frac{m_1m_2}{m_1+m_2}}_M |\der z|^2 \\
   &=\frac 12 M |\der z|^2
\end{align*}
The Lagrangian is 
\begin{align*}
 LM/2 &= \frac 12 M |\der z|^2+ \frac{Gm_1m_2}{|z|} \\
  L  &= |\der z|^2 + 2G\frac{m_1+m_2}{|z|} \\
  L &= |\der z|^2 + \frac {2\varg} {|z|}
\end{align*}
where $\varg =  G(m_1+m_2)$. 

By \nameref{th:cons-ang-mom}, $z\times \der z$ is constant. Suppose  $|z\times\der z|=\omega$. Without loss of generality, $z\in\C$. Furthermore, if $z=re^{i\phi}$, then on time interval $dt$,
\[
  \omega dt =z\times dz = r\times ird\phi=r^2d\phi
\]
So
\[
  d\phi = \frac{\omega}{r^2}dt
\]
Hence 
\begin{align*}
  |dz|^2&=|dr+ird\phi|^2=dr^2+\frac{\omega^2}{r^2}dt^2 \\
  |\der z|^2&= \der r^2 +\frac{\omega^2}{r^2}
\end{align*}
By conservation of energy,
\begin{align*}
  E &= \underbrace{\der r^2}_{T} +\underbrace{\frac{\omega^2}{r^2} -\frac {2\varg} r }_V\\
  \phi &= \int \frac{\omega dr/r^2}{\sqrt{E-V}} \\
  &=\int \frac{dr/r^2}{\sqrt{\frac{E}{\omega^2}+\frac{2g}{\omega^2}r^{-1}-r^{-2}}} \\
  &=\int \frac{dr/r^2}{\sqrt{ \underbrace{\frac{E}{\omega^2} +\frac{g^2}{\omega^4}}_{\smash{k^2}}-\left(r^{-1}-\frac{g}{\omega^2}\right)^2 }}
\end{align*} 
Substituting
\begin{align*}
  k\cos u = r^{-1}-\frac{g}{\omega^2}
  k\sin u du = dr/r^2
\end{align*}
gives 
\begin{align*}
  \phi &= \int \frac{k \sin u}{k\sin u} du = u  \\
  \sin\phi            &= \frac{\omega^2/r-g}{\sqrt{E\omega^2 +g}}+ C \\
  \sin\phi &= \frac{\omega^2/r}{\sqrt{E\omega^2+g}} +C
\end{align*}
Pick an origin for $\phi$ such that
\[
  \frac{\omega^2}r = 1+(\cos\phi)\sqrt{E\omega^2+g}
\]
This is a conic section with focus $0$ and eccentricity $e=\sqrt{E\omega^2+g}$, as the above implies $r/e=p-er\cos\phi$. 
\begin{center}
  \begin{asy}
    import graph;

    size(220,220);
    label("\fontspec{Noto Emoji}🌞",(0,0));
  
    real p = 1;
    real e = .6;
    real rad(real ang){
      return p/(1+e*cos(ang));
    }
    draw(polargraph(rad,0,2*pi),gray(.5));
    xequals(Label("$x=\frac p e$",position=0),p/e);
    
    real ang = 1;
    real x = rad(ang)*cos(ang);
    real y = rad(ang)*sin(ang);

    dot((x,y));
    draw("$r$",(0,0)--(x,y));
    draw("$\frac r e$",(x,y)--(p/e,y));
  \end{asy}
\end{center}

\end{document}

%%% Local Variables:
%%% mode: latex
%%% TeX-master: t
%%% End:

%%% Local Variables:
%%% mode: latex
%%% TeX-master: t
%%% End:
