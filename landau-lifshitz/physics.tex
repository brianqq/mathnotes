\documentclass{scrbook}

\usepackage[dropmargin]{brian}
\usepackage{mathtools}
\usepackage{bm}
%\usepackage{chngcntr}
\newcommand{\hyper}[1]{\prescript{*}{}{#1}}
\newcommand{\standard}[1]{\prescript{\sigma}{}{#1}}
\DeclareMathOperator{\st}{st}
\newcommand{\infeq}{\simeq}
\numberwithin{thms}{chapter}

\newcommand{\der}{\dot}
\newcommand{\dder}{\ddot}
\renewcommand{\vec}[1]{\bm{#1}}
\newcommand{\Der}{D}
 
\begin{document}
\chapter{Equations of motion}
\section{Lagrangian}
  \newcommand{\q}{\varq}
  \newcommand{\derq}{\der \q}
  \newcommand{\dt}{d\vart}
\begin{defn}[action]
  Let $\varL(\q,\derq, \vart)$ be a Lagrangian.
  Let $\q:\textrm{time}\to\textrm{configuration space}$. 
  Define action by 
  \[
    \varS(\q):=\int_a^b \varL(\q,\derq, \vart) \dt 
  \]
\end{defn}

Mechanics will minimize action of the appropriate Lagrangian.
\begin{theorem}[euler-lagrange]
  \label{th:euler-lagrange}
  \[
    \frac{d\varS}{d\q} = \inner{\frac{\partial \varL}{\partial \q} - \frac{d}{d\vart}\frac{\partial \varL}{d\derq}}{-}
  \]
\end{theorem}
\begin{proof}
  \newcommand{\dq}{h}
  \newcommand{\derdq}{h'}
  Let $\dq$ be an infinitesimal perturbation (endpoints are $0$). 
  \begin{align*}
    \varS(\q+\dq) &= \int \varL(\q+\dq,\derq + d\derq,t)\dt \\
           &= \int \varL(\q,\derq,\vart)\dt + \int \varD\varL(\varq,\derq, \vart)(\dq,d\derq,0) + o(\dq) \\
           &= \int \frac{\partial \varL}{\partial \q} \dq\dt + \underbrace{\frac{\partial L}{\partial \derq}}_{u} \underbrace{d\derq \dt}_{dv} \\
           &=\int\left(\frac{\partial \varL}{\partial \q} - \frac{d}{d\vart} \frac{\partial \varL}{\partial \derq}\right)\dq \dt + \underbrace{\left[\dq\frac{\partial L}{\partial \derq} \right]_a^b}_{\mathclap{0\textrm{ as $\dq(a)=\dq(b)=0$}}}
  \end{align*}
\end{proof}
\begin{cor}[stationary action]
  \[
    \frac{\partial \varL}{\partial \varq} - \frac{d}{d\vart}\frac{\partial \varL}{d\der \varq} = 0
  \]
\end{cor}

\chapter{conservation laws}
\section{energy}
\begin{defn}[energy]
  \label{def:energy}
  \[
    \varE\define  \inner{\Der \varq}{\frac{\partial\varL}{\partial \derq}} - L
  \]
\end{defn}
\begin{theorem}
  When $\Der(t\mapsto L(\_,\_,t)) =0$ (\eg closed system or constant field), energy is conserved. 
\end{theorem}
\begin{proof}
  By hypothesis, 
  \[
    \frac{d\varL}{d\vart} = \inner{\frac{\partial L}{\partial \varq}}{\der \varq} + \inner{\frac{\partial L}{\partial \derq}}{\dder \varq}
  \]
  By \nameref{th:euler-lagrange},
  \[
    \frac{d\varL}{d\vart} = \inner{\frac{d}{d\vart}\frac{\partial L}{\partial \der\varq}}{\der \varq} + \inner{\frac{\partial L}{\partial \derq}}{\dder \varq}
  \]
  By product rule, 
  \[
    \frac{d\varL}{d\vart} = \frac{d}{dt}\inner{\derq}{\frac{\partial L}{\partial \der \varq}}
  \]
  Subtracting $d\varL/d\vart$ finishes the proof. 
\end{proof}
\section{momentum}
\begin{defn}[momentum]\label{def:momentum}
  \renewcommand{\varv}{\vec v}
  \renewcommand{\varr}{\vec r}
  \[
    P\define \sum_a \frac{\partial \varL_a}{\partial\varv_a}
  \]
  where $a$ indexes over particles. 
\end{defn}
\begin{theorem}[conservation of momentum]
  In a closed system, $\der P=0$. 
\end{theorem}
\begin{proof}
  \newcommand{\nudge}{\vec\varepsilon}
  \newcommand{\dL}{dL}
  \renewcommand{\varr}{\vec r}
  This follows from homogeneity of space. Suppose every particle moves by infinitesimal $\nudge$ with velocities unchanged. Then 
  \[
    \dL = \sum_a \inner{\frac{\partial \varL}{\partial L}{\partial \varr_a}}{\nudge}
  \]
  which must be $0$ by homogeneity. As $\nudge$ is arbitrary, we conclude 
  \[
    \sum_a \frac{\partial \varL}{\partial \varr_a}=0
  \]
  By \nameref{th:euler-lagrange},
  \[
    \frac{d}{d\vart}\sum_a \frac{\partial \varL}{\partial \varv_a} = 0
  \]
\end{proof}
\begin{cor}[Newton's third law]
  \[
    \sum_a \vec \varF_a =0 
  \]
\end{cor}
\end{document}

%%% Local Variables:
%%% mode: latex
%%% TeX-master: t
%%% End:

%%% Local Variables:
%%% mode: latex
%%% TeX-master: t
%%% End:
