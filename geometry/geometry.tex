\documentclass{scrbook}
\usepackage{brian}
\usepackage[inline]{asymptote}

\renewcommand{\proofSymbol}{\textrm{\fontspec{NotoEmoji-Regular}👍}}

\usepackage[style=alphabetic,backend=bibtex]{biblatex}
\bibliography{geometry}



\begin{document}
\chapter[Axioms]{axioms}
\section[Euclid's axioms]{euclid's axioms}
\begin{enumerate}
\item There is one line through any two distinct points.
\item There is one line containing any nondegenerate line segment.
\item For any point $p\in\R^2$ and any radius $r\in\R$, there is exactly one circle with center $p$ and radius $r$. 
\item All right angles are equal.   
\item Suppose two lines cross an axis. If, on one side of the axis, the interior angles sum to under $\pi/2$, then the lines cross on that side.
  \begin{figure}[h!]
    \centering
    \caption{$\alpha+\beta <\pi/2$}
    \begin{asy}
      size(200);
      path axis = (0,0)--(0,80);
      path top =(-20,10)--(110,45);
      path btm = (-20,70)--(110,35);
      draw(axis);
      draw(top);
      draw(btm);
      label("$\alpha$",point(axis,intersectionpoint(axis,top)),NE);
      label("$\beta$",point(axis,intersectionpoint(axis,btm)[0]),SE);
    \end{asy}
  \end{figure}
\end{enumerate}
\begin{defn}[parallel postulate]
  For a line $L$ and point $p\notin L$, there is one line through $p$ parallel to $L$. 
\end{defn}
This ensures we are studying the Euclidean plane, \ie $\R^2$. 
\section[Constructions]{constructions}
\begin{theorem}
  Every line segment is the leg of some equilateral triangle. 
\end{theorem}
\begin{proof}
  Start with the line segment:
  \begin{center}
    \begin{asy}
      dot((0,0));
      dot((30,0));
      draw((0,0)--(30,0));
    \end{asy}
  \end{center}
  \begin{center}
    \begin{asy}
      import graph; 
      path c1=Circle((0,0),30);
      path c2=Circle((30,0),30);
      draw(c1,blue);
      draw(c2,blue);
      dot((0,0));
      dot((30,0));
      pair p = intersectionpoint(c1,c2);
      dot(p);
      draw((0,0)--p--(30,0)--cycle);
    \end{asy}
  \end{center}
\end{proof}
\end{document}

%%% Local Variables:
%%% mode: latex
%%% TeX-master: t
%%% End:
