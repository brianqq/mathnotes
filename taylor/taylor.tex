\documentclass[10pt]{scrartcl}
\usepackage[dropmargin]{brian}
%\usepackage{cfr-lm}
%\usepackage{lmodern}
\usepackage{float}

\usepackage[style=alphabetic,backend=bibtex]{biblatex}
\bibliography{taylor}

% proof environmnet definition 
\renewcommand{\ref}{\hyperref}
\newcommand{\E}{\mathbb E}

%categories
\newcommand{\cattop}{\mathtt{top}}
\newcommand{\ab}{\mathtt{Ab}}


% \AtBeginDocument{
%   \pdfhorigin=\dimexpr\pdfhorigin-1.5in\relax
%   \pdfvorigin=\dimexpr\pdfvorigin-1in\relax
%   \pdfpagewidth=\dimexpr\pdfpagewidth-3in\relax
%   \pdfpageheight=\dimexpr\pdfpageheight-2in\relax
% }
\numberwithin{thms}{section}
\begin{document}
\section{taylor series}
\newcommand{\taylor}{\mathtt{taylor}}
\begin{lemma}[mean value theorem: integration]
  \label{mvt-int}
  If $f:[a,b]\to_\cattop\R$ and $\phi:[a,b]\to_\cattop[0,\infty)$ then there is some $x:[a,b]$ s.t.
  \[
  \int_a^bf(t)\phi(t)=f(x)\int_a^b\phi(t)dt
  \]
  \cite{wiki:mvt-int}
\end{lemma}
\begin{proof}
  Consider the case where $\int_a^b\phi=1$. Then it suffices to show for some $x:[a,b]$, we have $f(x)=\E(f;d\phi)$. By compactness, we can find some 
  \newcommand{\xmax}{x_{\mathtt{max}}}
  \newcommand{\xmin}{x_{\mathtt{min}}}
  $\xmax:[a,b]$ maximizing $f$ and $\xmin:[a,b]$ minimizing $f$. As $\E(f;d\phi):[f(\xmin),f(\xmax)]$, connectedness gives the desired $x$. 
\end{proof}

\begin{defn}[Taylor polynomial]
  Given $f:C^N(\R;\R)$, define its $N$-degree Taylor polynomial centered at $x_0$ as
  \[
  \taylor_{x_0}^Nf(x)\define\sum_{n=0}^N D^nf(x_0) \frac{(x-x_0)^n}{n!}
  \]
\end{defn}

% \begin{theorem}[Cauchy repeated integration]
%   \label{th:cauchy-rep-int}
%   Suppose $f:(\R\to\R)_\cattop$. Let
%   \[
%   D^{-n}f(x)=\int_{x_0}^x\int_{x_0}^{\sigma_1}\dots\int_{x_0}^{\sigma_{n-1}}f(\sigma_n)d\sigma_n\dots d\sigma_1
%   \]
%   Then
%   \[
%   D^{-n}f(x) = \frac 1 {(n-1)!} \int_{x_0}^x(x-t)^{n-1}f(t)dt
%   \]
%   \cite{wiki:cauchy-rep-int}
% \end{theorem}
% \begin{proof}
  
% \end{proof}

\begin{theorem}[Taylor's theorem: Lagrange remainder]
  \label{taylor-thm-lagrange}
  For $f:C^{N+1}(\R;\R)$, 
  \[
  f(x)=\taylor^Nf(x)+\frac 1{N!}\int_{x_0}^xD^{N+1}f(t)(x-t)^Ndt
  \]
  and for some $x^*:(x_0,x)$,  
  \[
  f(x)=\taylor^Nf(x)+D^{N+1}f(x^*)\frac{(x-x^*)^{N+1}}{(N+1)!}
  \]
\end{theorem}
\begin{proof}
  Suppose $f:C^{N+1}(\R;\R)$. 
  By the fundamental theorem of calculus,
  \[
  f(x)=f(x_0)+\int_{x_0}^x Df(t)dt=f(x_0)+\int_{x_0}^x\underbrace{D^1f(t)}_{u}\underbrace{(x-t)^0dt}_{dv}
  \]
  Then $du=D^2f(t)$ and $v=-(x-t)$. 
  Integrating by parts gives
  \begin{align*}
    f(x)&=f(x_0)-Df(t)(x-t)\biggl|_{t=x_0}^x+\int_{x_0}^xD^2f(t)(x-t)dt\\
        &=f(x_0)+Df(x_0)(x-x_0)+\int_{x_0}^x\underbrace{D^2f(t)}_u\underbrace{(x-t)dt}_{dv}\\
        &=f(x_0)+\dots+Df(x_0)\frac{(x-x_0)^{n-1}}{(n-1)!} + \frac 1 {(n-1)!}\int_{x_0}^x \underbrace{D^nf(t)}_{u}\underbrace{(x-t)^{n-1}dt}_{dv}\\
    f(x)&=\sum_{n=0}^{N}D^nf(x_0)\frac{(x-x_0)^n}{n!}+\frac 1 {N!}\int_{x_0}^xD^{N+1}f(t)(x-t)^{N}dt
  \end{align*}
The $x^*$ is given by \nameref{mvt-int}.
\end{proof}
\begin{cor}[Cauchy's formula for repeated integration]
  \[
  D^{-n}f(x)=\frac 1 {(n-1)!}\int_{x_0}^xf(t)(x-t)^{n-1}dt
  \]
  \cite{wiki:cauchy-rep-int}
\end{cor}

\begin{cor}[Taylor's theorem: no remainder]
  For $f:C^N(\R;\R)$, 
  \[
  f(x_0+h)= \taylor^Nf(x_0+h)+o(h^N)
  \]
\end{cor}
\begin{proof}
  By \nameref{taylor-thm-lagrange}, 
  \[
  f(x_0+h)=\taylor^{N-1}f(x_0+h)+D^Nf(x^*)\frac{(x^*-x_0)^N}{N!}
  \]
  Suppose $h$ is infinitesimal. Then $x^*=x_0+h^*$, where $h^*:[0,h]$. By continuity,
  \[
  D^Nf(x_0+h^*)\frac{(h^*)^N}{N!}= D^Nf(x_0)\frac{(h^*)^N}{N!}+O(h^*)\frac{(h^*)^N}{N!}
  \]
  So
  \[
  D^Nf(x^*)= D^Nf(x_0)+O(h^{N+1})
  \]
\end{proof}

\section{analytic functions}
\begin{defn}
  A function $f:C^\infty(\R;\R)$ is analytic in an open interval $(a,b)$ iff $f|(a,b)=\taylor^\infty f|(a,b)$. The function is analytic on $\R$ iff for each $x:\R$, there is some open interval containing $x$ on which $f$ is analytic. The space of analytic functions is denoted $C^\omega(\R;\R)$. 
\end{defn}

\begin{defn}
  A function in $C^\omega(\C;\C)$ is called entire. 
\end{defn}

\begin{lemma}[ratio test]
  \label{ratio-test}
  Let $a_n:\C$ be a sequence. If
  \begin{equation}
    \label{eq:ratio-test-hyp}
    \lim_{n\to \infty} \left|\frac{a_{n+1}}{a_n}\right| <1
  \end{equation}
  then $\sum_{n=0}^\infty a_n$ converges absolutely.
  \cite{wiki:ratio-test}
\end{lemma}
\begin{proof}
  (Adapted from \cite{wiki:ratio-test})
  For absolute convergence, it suffices to consider the case where $a_n = |a_n|:[0,\infty)$. Suppose \cref{eq:ratio-test-hyp} holds. Let
  \begin{equation}
    \label{eq:ratio-test-r-def}
  r\define \lim_{n\to \infty} \frac{\frac{a_{n+1}}{a_n} +1}{2}<1
  \end{equation}
  For some $N$, for all $n\geq N$, \cref{eq:ratio-test-r-def} gives $a_{n+1}<r a_n$. But then $a_{n}<r^{n-N}a_N$, so
  \[
  \sum_{n=N}^\infty a_n \leq \sum_{n=N}^\infty r^{n-N}a_N\to 0
  \]
  as $r<1$. 
\end{proof}

\begin{defn}[$\exp$]
  Let
  \[
  \exp(x) \define \sum_{n=0}^\infty \frac{x^n}{n!}
  \]
  By \nameref{ratio-test}
  \[
  \left |\frac {x^{n+1}}{(n+1)!}\frac{n!}{x^n} \right|=\left|\frac x {n+1} \right|\to 0
  \]
  Hence $\exp:\C\to\C$ is well-defined and entire.  
\end{defn}

\begin{lemma}[$\exp$ homeomorphism]
  \[
  \exp(x+y)=\exp(x)\exp(y)
  \]
\end{lemma}
\begin{proof}
  \begin{align*}
    \exp(x+y)&=\sum_{n=0}^\infty \frac{(x+y)^n}{n!}
             =\sum_{n=0}^\infty  \sum_{k=1}^n  {n \choose k} \frac{x^{n-k}y^k}{n!} \\
             &=\sum_{n=0}^\infty  \sum_{k=1}^n  \frac{x^{n-k}y^k}{(n-k)!k!} \\ 
    \exp(x)\exp(y)&=\left(\sum_{\alpha=0}^\infty \frac{x^\alpha}{\alpha!} \right)\left(\sum_{\beta=0}^\infty \frac{y^\beta}{\beta!}\right) =\sum_{\alpha=0}^\infty\sum_{\beta=0}^\infty \frac{x^\alpha}{\alpha!}\frac{y^\beta}{\beta!}
  \end{align*}
  with equivalence given by setting $\alpha=n-k$ and $\beta=k$.
\end{proof}
\begin{cor}
  $\exp(xn)=[\exp(x)]^n$
\end{cor}

\begin{lemma}
$\left(\exp(it)\right)^*=\exp(-it)$ 
\end{lemma}
\begin{proof}
  \begin{align*}
    \exp(it)&=\sum_{n=0}^\infty\frac{(it)^{2n}}{(2n)!} +\frac{(it)^{2n+1}}{(2n+1)!}\\
            &=\sum_{n=0}^\infty\frac{(-1)^n(t)^{2n}}{(2n)!} +i\frac{(-1)^n(t)^{2n+1}}{(2n+1)!}\\
    \exp(-it) &=\sum_{n=0}^\infty\frac{(-1)^nt^{2n}}{(2n)!} -i\frac{(-1)^nt^{2n+1}}{(2n+1)!}\\
\end{align*}
\end{proof}

\begin{theorem}[Euler's theorem]
  The map $\exp(i-):\R\to\C$ is a universal covering map of the unit circle. In particular, $\exp(it)$ is the point on the unit circle $t$ radians counterclockwise from 1. 
\end{theorem}
\begin{proof}
  As $|\exp(it)|=\exp(it)\exp(-it)=1$, its image is contained in $S=\{z:|z|=1\}$.
  Note $D\exp(it)=i\exp(it)$, so $\exp(it)$ moves clockwise. 
  Finally, consider arc length.
  \[
  s(t)= \int_0^t |D\exp(it)|dt = \int_0^t |i\exp(it)|dt = \int_0^t dt = t
  \]
\end{proof}

\begin{example}[hyperbolic trig]
  \begin{align*}
    \sinh(z)&=\frac{e^z-e^{-z}}2\\
    \cosh(z)&=\frac{e^z+e^{-z}}2\\
    \tanh(z)&=\frac{\sinh(z)}{\cosh(z)}=\frac{e^z-e^{-z}}{e^z+e^{-z}}
  \end{align*}
  They are entire.
\end{example}

\begin{example}[trig]
  \begin{align*}
    \sin(z)&=-i\sinh(iz)\\
    \cos(z)&=\cosh(iz)\\
    \tan(z)&=\frac{\sin z}{\cos z}
  \end{align*}
  They are entire. 
  Note that for $x:\R$, 
  \begin{align*}
    \sin(x)&=\Re \exp(ix)\\
    \cos(x)&=\Im \exp(ix)
  \end{align*}
\end{example}

\begin{defn}[$\ln$]
  Let $\ln:[0,\infty)\to_\cattop\R$ be the inverse of $\exp$. The inverse function theorem guarantees its existence.
\end{defn}

\begin{theorem}
  The map $\ln (x)$ is analytic on $(0,2)$. 
\end{theorem}
\begin{proof}
  By the inverse function theorem, 
  \begin{align*}
    D\ln(x)&=x^{-1} \\
    D^2\ln(x)&=-x^{-2}\\
    D^n\ln(x)&=(-1)^{n+1}(n-1)!x^{-n}\\
  \end{align*}
  So 
  \[
  \taylor_1^\infty\ln(x)=\sum_{n=1}^\infty (-1)^{n+1}\frac{(x-1)^{n}}{n}
  \]
  But \nameref{ratio-test} gives 
  \[
  \left|\frac{(x-1)^{n+1}}{n+1}\frac{n}{(x-1)^n}\right|=|x-1|\frac{n}{n+1}\to |x-1|
  \]
  so $\taylor_1^\infty\ln(x)$ converges for $x:(0,2)$.

  By \nameref{taylor-thm-lagrange}, for some $t$ between $1$ and $x$, 
  \begin{align*}
    \ln(x)-\taylor_1^n\ln(x) &= \frac{(-1)^{n+1}t^{-n-1}(x-t)^{n+1}}{n+1}\\
    |\ln(x)-\taylor_1^n\ln(x)| &= \left| \frac{t^{-n-1}(x-t)^{n+1}}{n+1}\right|\\
                             &\leq \left|\frac{\max(1,x^{-n-1})(x-1)^{n+1}}{n+1}\right|\to 0
  \end{align*}
  when $x:(0,2)$. 
\end{proof}

\begin{cor}
  The map $\sqrt{x}$ is analytic on $(0,2)$. 
\end{cor}
\begin{proof}
  \[
  \sqrt{x}=\sqrt{\exp\ln x}=\exp\left(\frac 1 2 \ln x\right)
  \]
\end{proof}
% \begin{proof}
\begin{example}[$\taylor \sqrt{x}$]
  \begin{align*}
    D(x)^{1/2}&=\frac 1 2 x^{-1/2}\\
    D^2(x)^{1/2}&=-\frac 1 4 x^{-3/2}\\
    D^n(x)^{1/2}&=(-1)^{n}\frac{(2n-1)!! x^{1/2-n}}{(1-2n)2^n}
  \end{align*}
  Thus
  \[
  \taylor_1^\infty \sqrt{x}=\sum_{n=0}^\infty \frac{(-1)^{n}
    (2n-1)!!(x-1)^n}{(1-2n)2^nn!}
  \]
\end{example}
%But
% \[
% \frac{(2n+1)!!(x-x_0)^{n+1}}{2^{n+1} (n+1)!}\frac{2^n n!}{(2n-1)!!(x-1)^n}=\frac{2n+1}{2n}(x-1)^n\to (x-1)^n
% \]
% so $\taylor_1^\infty \sqrt{x}$ converges on $(0,2)$. 

% By \nameref{taylor-thm-lagrange}, for some $t$ between $x$ and $1$, 
% \begin{align*}
%   \sqrt x-\taylor_1^N\sqrt x &= \frac{D^{N+1}\sqrt{t}(x-t)^{N+1}}{2^{N+1}(N+1)!}\\
%   |\sqrt x-\taylor_1^N\sqrt x| &= \left|\frac{(2N+1)!!t^{-N-1/2}(x-t)^{N+1}}{2^{N+1}(N+1)!}\right|\\
%                              &\leq \left|\frac{(2N+1)!!(x-1)^{N+1}}{2^NN!}\right|
% \end{align*}
% As $(2N+1)!!=(2N+1)(2N-1)\dots 1 =O(2^{N/2})$, this goes to $0$. Hence, for $x:(0,2)$ we find $\sqrt{x}=\taylor^\infty_1\sqrt{x}$. 
% \end{proof}

\begin{lemma}[double factorial]
  \begin{equation}
    (2n-1)!!=\frac{(2n)!}{2^nn!}\label{eq:!!fact}
  \end{equation}
\end{lemma}
\begin{proof}
  Base case: $n=1$. Note $1!!=1$. Likewise, note $2!/2=1$. Hence \cref{eq:!!fact} holds for $n=1$. 

  Suppose for the sake of induction that \cref{eq:!!fact} holds. Note
  $(2n+1)!!=(2n+1)(2n-1)!!$. Consider 
  \[
  \frac{(2n+2)!}{2^{n+1}(n+1)!}=\frac{(2n+2)(2n+1)(2n)!}{2(2^n)(n+1)n!}=(2n+1)n!!
  \]
\end{proof}
\begin{cor}
  \[
  \taylor_1^\infty\sqrt x= \sum_{n=1}^\infty\frac{(-1)^n(2n)!(x-1)^n}{(1-2n)(n!)^2(4^n)}
  \]
  as in \cite{wiki:sqrt}.
\end{cor}
\section{power series}
\begin{theorem}[radius of convergence]
  Let $P$ be a power series, $a:\C$. If $P(a)$ converges, then for all $z:\C$ such that $|z|<|a|$, we have $P(z)$ converges absolutely \cite{needham:visual-complex}.
\end{theorem}
\begin{proof}
  Let $P(z)=c_0+c_1z^1+c_2z^2\dots$. By hypothesis, $P(a)=c_0+c_1a+c_2a^2+\dots$ converges. Then $c_na^n\to 0$, so we can find some $M$ such that $|c_na^n|\leq M$ for all $n$. For $|z|<|a|$, thus $|z|/|a|<1$, 
  \[
  \sum_{n=N}^\infty |c_n z^n|=\sum_{n=N}^\infty |c_n||a|^n \frac{|z|^n}{|a|^n} \leq M\sum_{n=N}^\infty \frac{|z|^n}{|a|^n} =\dfrac{M\dfrac{|z|^N}{|a|^N}}{1-\frac{|z|}{|a|}}\to 0
  \]
  Hence $P(z)$ converges absolutely.
  \cite{needham:visual-complex}. 
\end{proof}
\begin{cor}
  If $P(a)$ diverges, then $P(z)$ diverges for all $z:\C$ such that $|z|>|d|$
\cite{needham:visual-complex}. 
\end{cor}


\begin{theorem}[identity theorem]
  Let $z_n:\C$ be a sequence such that $z_n\to 0$. Let $P$ and $Q$ be power series. If $P(z_n)=Q(z_n)$, then $P=Q$. 
  \cite{needham:visual-complex}. 
\end{theorem}
\begin{proof}
  Let $P(z)=a_0+a_1z+a_2z^2\dots$ and $Q(z)=b_0+b_1z+b_2z^n\dots$.
  \begin{equation}
    \label{eq:id-proof-equal-pows}
    P(z_n)=a_0+a_1z_n+a_2z_n^2=b_0+b_1z_n+b_2z_n^2\dots = Q(z_n)
  \end{equation}
  Then as $\lim P(z_n)=\lim Q(z_n)$, we have $a_0=b_0$. Hence by \cref{eq:id-proof-equal-pows}, we get
  \begin{align*}
    a_1z_n+a_2z_n^2\dots &= b_1z+b_2z_n^2\dots\\
    a_1+a_2z_n^1+a_3z^2\dots &= b_1+b_2z_n^1+b_3z_n^2\dots\\
  \end{align*}
  Taking a limit gives $a_1=b_1$. Repeating this inductively gives $P=Q$. 
\end{proof}

\section{non-analytic smooth functions}
\begin{counterex}
  \[
  f(x)\define
  \begin{cases}
    \exp(-x^{-1}),&x>0\\
    0,&x\leq 0
  \end{cases}
  \]
  Smoothness follows from noting
  \begin{align*}
    D^nf(0)&=\lim_{x\to 0}\frac{f(x)}{x^n}=\frac{\exp(-x^{-1})}{x^n}=\lim_{x\to 0}\frac{x^{-n}}{\exp(x^{-1})}\\
           &=\lim_{x\to 0}\frac{x^{-n}}{1+x^{-1}+x^{-2}/2+\dots x^{-n}/n!\dots}=0
  \end{align*}
  with, 
  \[
  \taylor_0^\infty f(x)=0
  \]
  hence $f$ is not analytic at $0$. 
  \cite{wiki:smooth-non-analytic}
\end{counterex}


\printbibliography
\end{document}
%%% Local Variables:
%%% mode: latex
%%% TeX-master: t
%%% End:
