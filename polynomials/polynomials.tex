\documentclass[10pt]{article}
\usepackage{amssymb, amsmath}
\usepackage[amsmath,thmmarks,amsthm]{ntheorem}
\usepackage{thmtools}
%\usepackage{cfr-lm}
\usepackage{lmodern}
\usepackage{float}
\usepackage{url}
\usepackage{microtype}

\usepackage{polynom}

\usepackage[style=alphabetic,backend=bibtex]{biblatex}
\bibliography{polynomials}

\usepackage{nameref}

\usepackage[colorlinks=true,linkcolor=blue,citecolor=blue,urlcolor=blue]{hyperref}
\renewcommand{\ref}{\hyperref}

\usepackage{cleveref}

\newcounter{thms}
%\numberwithin{thms}{section}
\theoremstyle{definition}
\newtheorem{defn}[thms]{Definition}
\newtheorem{defcat}[thms]{Category definition}
\newtheorem{example}[thms]{Example}
\theoremstyle{definition}
\newtheorem{counterex}[thms]{Counter example}

\theoremstyle{plain}
\newtheorem{theorem}[thms]{Theorem}
\newtheorem{lemma}[thms]{Lemma}
\newtheorem{cor}[thms]{Corollary}

\crefname{lemma}{lemma}{lemmas}
\crefname{theorem}{theorem}{theorems}

\newcommand{\Z}{\mathbb{Z}}
\newcommand{\N}{\mathbb{N}}
\newcommand{\R}{\mathbb{R}}
\newcommand{\C}{\mathbb{C}}
\newcommand{\E}{\mathbb{E}}

\newcommand{\define}{:\equiv}
\DeclareMathOperator{\cl}{cl}
\DeclareMathOperator{\sgn}{sgn}

%categories
\newcommand{\cattop}{\mathtt{top}}
\newcommand{\ab}{\mathtt{Ab}}

\AtBeginDocument{
  \pdfhorigin=\dimexpr\pdfhorigin-1.5in\relax
  \pdfvorigin=\dimexpr\pdfvorigin-1in\relax
  \pdfpagewidth=\dimexpr\pdfpagewidth-3in\relax
  \pdfpageheight=\dimexpr\pdfpageheight-2in\relax
}

\begin{document}

\begin{theorem}[polynomial remainder theorem]
  Suppose $P(x):\C[x]$. Then 
  \[
  P(x)=Q(x)(x-k)+P(k)
  \]
  In particular, $(x-k)|P(x)$ iff $P(k)=0$
\end{theorem}
\begin{proof}
  By synthetic division, 
  \[
  \begin{array}{llllll}
    a_n & a_{n-1} & a_{n-2}             & \dots & a_0 \\
         & ka_n   &  a_{n-1}k+a_nk^2    &\dots & ka_1+\dots k^na_n\\
    \hline
     a_n & a_{n-1}+ka_n & a_{n-2}+ a_{n-1}k+a_nk^2& \dots & a_0+ka_1+\dots k^na_n
  \end{array}
  \]
\end{proof}


\begin{defn}[Vandermonde matrix]
  \[
  V\define
  \begin{bmatrix}
    1 & x_0 & x_0^2&\dots&x_0^n \\
    1 & x_1 & x_1^2&\dots&x_1^n \\
    \vdots &\vdots & \vdots &\ddots&\vdots \\
    1 & x_n & x_n^2&\dots&x_n^n 
  \end{bmatrix}
  \]
  \cite{wiki:vandermonde}
\end{defn}

\begin{lemma}[Vandermonde determinant]
  \label{vandermonde-det}
  \[\det V = \prod_{1\leq i < j \leq n} (a_j-a_i)\]
\end{lemma}
\begin{proof}
  Consider
  \allowdisplaybreaks[2]
  \begin{align*}
    |V_n|&=
           \begin{vmatrix}
             1 & x_0 & x_0^2&\dots&x_0^n \\
             1 & x_1 & x_1^2&\dots&x_1^n \\
             \vdots &\vdots & \vdots &\ddots&\vdots \\
             1 & x_n & x_n^2&\dots&x_n^n 
           \end{vmatrix}\\
         &=
           \begin{vmatrix}
             1 & x_0 & x_0^2&\dots&x_0^n \\
             0 & x_1-x_0 & x_1^2-x_0^2&\dots&x_1^n-x_0^n \\
             \vdots &\vdots & \vdots &\ddots&\vdots \\
             0 & x_n-x_0 & x_n^2-x_0^2&\dots&x_n^n -x_0^n
           \end{vmatrix}\\
         &=
            \begin{vmatrix}
             1 & x_0 & x_0^2&\dots&0 \\
             0 & x_1-x_0 & x_1^2-x_0^2&\dots&x_1^n-x_0x_1^{n-1} \\
             \vdots &\vdots & \vdots &\ddots&\vdots \\
             0 & x_n-x_0 & x_n^2-x_0^2&\dots&x_n^n -x_0x_n^{n-1}
           \end{vmatrix}\\
          &= 
            \begin{vmatrix}
              1 & 0 & 0 &\dots&0 \\
              0 & x_1-x_0 & (x_1-x_0)x_1 & \dots & (x_1-x_0)x_1^{n-1} \\
              \vdots &\vdots & \vdots &\ddots&\vdots \\
             0 & x_n-x_0 & (x_n-x_0)x_n&\dots&(x_n-x_0)x_n^{n-1}
           \end{vmatrix}\\
         &= (x_n -x_0)\dots(x_1-x_0) 
            \begin{vmatrix}
              1 & x_1 & \dots & x_1^{n-1} \\
              \vdots &\vdots &\ddots&\vdots \\
              1 & x_n&\dots&x_n^{n-1}
           \end{vmatrix}\\
         &=(x_n-x_0)\dots (x_1-x_0)(x_n-x_1)\dots (x_n-x_{n-1})\dots
  \end{align*}
  \cite{proofwiki:vandermonde}
\end{proof}

\begin{theorem}[unisolvence theorem]
  The $n+1$ points $(x_0,y_0)\dots (x_n,y_n)$ with distinct $x_i$ determine a unique $n$-degree polynomial. 
\end{theorem}
\begin{proof}
  \begin{equation}
  \begin{bmatrix}
    1 & x_0 & x_0^2&\dots&x_0^n \\
    1 & x_1 & x_1^2&\dots&x_1^n \\
    \vdots &\vdots & \vdots &\ddots&\vdots \\
    1 & x_n & x_n^2&\dots&x_n^n
  \end{bmatrix}
  \begin{bmatrix}
    a_0 \\
    a_1 \\
    \vdots \\
    a_n
  \end{bmatrix}
  =
  \begin{bmatrix}
    y_0\\
    y_1 \\
    \vdots \\
    y_n
  \end{bmatrix}
\label{eq:poly-solve}
\end{equation}
  By \nameref{vandermonde-det}, $V$ is isomorphic, as each $x_i$ is distinct. Hence (\ref{eq:poly-solve}) has a unique solution. 
\end{proof}


\printbibliography
\end{document}
%%% Local Variables:
%%% mode: latex
%%% TeX-master: t
%%% End:

%%% Local Variables:
%%% mode: latex
%%% TeX-master: t
%%% End:
