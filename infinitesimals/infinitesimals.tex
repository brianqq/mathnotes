\documentclass{scrartcl}
\usepackage{brian}
\renewcommand{\proofSymbol}{\ensuremath{\text{\fontspec{NotoEmoji-Regular}👍}}}

\title{Real infinitesimals}
\begin{document}
\section{Why?}
When first learning calculus, I thought in terms of infinitesimal rates of change. I found this intuitive, thought provoking, and effective. There were many holes in my understanding. In particular, I viewed infinitesimals as like zero but not really. Not quite mathematical precision. 

Newton and Leibniz conceived calculus aided by the notion of the infinitesimal. Leibniz assumed there were infinitesimal quantities that obeyed the same rules as the real numbers. To take a derivative, say of $y=x^2$, you would calculate the infinitesimal rate of change $dy/dx$ where $dy$ and $dx$ are infinitesimals: 
\begin{align*}
  \frac{dy}{dx} &= \frac{(x+dx)^2-x^2}{dx} \\
                &= \frac{x^2+2xdx+dx^2-x^2}{dx} \\
                &= \frac{2xdx+dx^2}{dx} \\
                &= 2x+dx
\end{align*}
and then you discard the infinitesimal, giving the correct answer. 

This argument only works if $dx\neq 0$. But then how can you discard the $dx$? Calculus stood on embarrassingly shaky ground for nearly 300 years. Critics derided infinitesimals as `the ghosts of departed quantities'. Cauchy and Weierstrass eventually settled the matter by introducing the modern notion of limit, and infinitesimals were relegated to the back of a napkin. 

Infinitesimals in calculus are not merely a historical curiosity or a useful heuristic. They are a fully rigorous mathematical technique. They have been used to prove novel theorems. Infinitesimals have fewer moving parts than $\varepsilon$-$\delta$ limits, so they simplify many arguments. 

\section{What?}
We will enlarge $\R$ to the hyperreals, $*\R$. This contains all real numbers, as well as an infinite number $\omega$ that is larger than all standard reals. Likewise, it contains the infinitesimal $1/\omega$. We want it to obey the same laws as $*\R$. 

We will accomplish this by showing that the axioms defining such a $*\R$ are not contradictory.
\section{How?}
Start with the axioms for $\R$. It contains the constants $0$ and $1$. We can take addition $+$, and multiplication $\bullet$. There is an order $\leq$ on $\R$.  

They are governed by the axioms, where we suppose $x,y,z$ can be any real number.
\begin{enumerate}
\item Additive identity:  $x+0=x$
\item Additive inverses exist
\item $+$ is associative
\item Multiplicative identity: $1x=x$.
\item Multiplication is associative: $xy =yx$. 
\item Multiplication distributes: $x(y+z)=xy+yz$. 
\item If $x < y$ then $x+z < y+ z$
\item If $x > 0$ and $y > 0$ then $xy > 0$. 
\item Completeness: any subset $S$ of $R$ with an upper bound has a least upper bound. 
\end{enumerate}

\begin{theorem}
  We extend $\R$ to $*\R$ by adding some infinite number $\omega$ and augmenting the list of axioms with
  \begin{align*}
    &\omega \geq 1\\
    &\omega \geq 2\\
    &\omega \geq 3\\
    &\vdots
  \end{align*}
  This axiomatic system is free from contradictions
\end{theorem}
\begin{proof}
  Suppose we can prove a contradiction starting from these axioms. A proof has finitely many steps, so it can only use finitely many axioms, hence only finitely many of our new axioms. But the standard real numbers $\R$ with $\omega=n$ satisfies the first $n$ of our new axioms. This means we can prove a contradiction from the axioms of $\R$, which is impossible. 
\end{proof}
\section{Comparison}
A standard proof that $f(x)=x^2$ is continuous:
\begin{proof}
  Let
  \[
    h < \min\left(\frac{\sqrt\varepsilon}{2(1+2x)},1\right)
  \]
  Then 
  \begin{align*}
    |f(x+h)-f(x)| &= \left| (x+h)^2 - x^2\right| \\
                  &= \left| x^2+2xh+h^2-x^2\right| \\
                  &= \left| 2xh+h^2\right| \\
                  &< \left| 2x\frac{\sqrt\varepsilon}{2(1+2x)} + \frac{\varepsilon}{(2(1+2x))^2} \right|\\
                  &\leq \varepsilon
  \end{align*}
\end{proof}
A nonstandard proof:
\begin{proof}
  Let $x\approx y$ iff $x-y$ is infinitesimal. 
  \begin{align*}
    f(x+dx)&=x^2+2xdx+dx^2\\
           &=f(x)+2xdx+dx^2
  \end{align*}
  any finite number times an infinitesimal is infinitesimal. Hence $2xdx+dx^2\approx 0$. 
  Then 
  \[
    f(x+dx)\approx f(x)
  \]
\end{proof}
\end{document}

%%% Local Variables:
%%% mode: latex
%%% TeX-master: t
%%% End:
