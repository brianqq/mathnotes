\documentclass{scrartcl}
\usepackage{amssymb, amsmath}
\usepackage[amsmath,thmmarks]{ntheorem}
\usepackage{microtype}

\usepackage{fontspec}
\setmainfont{Linux Libertine O}
\setmonofont{Inconsolatazi4}
\usepackage[colorlinks=true,linkcolor=blue,citecolor=blue,urlcolor=blue]{hyperref}
\renewcommand{\ref}{\hyperref}
\usepackage{cleveref}
%proof
\makeatletter
\newcommand{\openbox}{\leavevmode
  \hbox to.77778em{%
  \hfil\vrule
  \vbox to.675em{\hrule width.6em\vfil\hrule}%
  \vrule\hfil}}
\gdef\proofSymbol{\openbox}
\newcommand{\proofname}{Proof.}
\newcounter{proof}\newcounter{currproofctr}\newcounter{endproofctr}%
\newenvironment{proof}[1][\proofname]{
  \th@nonumberplain
  \def\theorem@headerfont{\ttfamily}%
  \normalfont
  %\theoremsymbol{\ensuremath{_\blacksquare}}
  \@thm{proof}{proof}{#1}}%
  {\@endtheorem}
\makeatother

\theoremstyle{definition} 
\theoremheaderfont{\ttfamily}
\theorembodyfont{\normalfont}


\usepackage{euler}
\DeclareMathOperator{\Span}{span}
\newcommand{\R}{\mathbb{R}}
\newcommand{\PP}{\mathbb{P}}
\newcommand{\B}{\mathbb{B}}
\newcommand{\Z}{\mathbb{Z}}
\newcommand{\N}{\mathbb{N}}
\newcommand{\ie}{\emph i.e.}


\newcommand{\norm}[1]{|#1|}
\newcommand{\normsq}[1]{\norm{#1}^2}
\newcommand{\inner}[1]{\langle #1\rangle}
\newtheorem{theorem}{Theorem}
\newtheorem{corr}{Corollary}
\newtheorem{lemma}{Lemma}
\newtheorem{definition}{Definition}
\begin{document}
Let $X$ be a normed space space over $\R$ and suppose the parallelogram law holds, \ie 
\begin{equation}
  \normsq{x+y}+\normsq{x-y}=2\normsq{x}+2\normsq{y}
  \label{parid}
\end{equation}
for all $x,y\in X$. This is a necessary condition for $X$ to be an inner product space. I will show it is also a sufficient condition

Let $\inner{x,y}:=\normsq{x+y}-\normsq x - \normsq y$.
Note that $\inner{x,y}=\inner{y,x}$, so this is symmetric.
Note also that $\inner{x,y}$ is continuous as a map from $X^2\to \R$.
\begin{lemma}
$\inner{-x,y}=-\inner{x,y}$
\end{lemma}
\begin{proof}
  Observe that 
  \begin{align*}
    \inner{x,y}&=\normsq{x+y}-\normsq x - \normsq y\\
    \inner{-x,y}&=\normsq{-x+y}-\normsq x - \normsq y
  \end{align*}
  So 
  \[
  \inner{x,y}+\inner{-x,y}=\normsq{x+y}+\normsq{x-y}-2\normsq x - 2\normsq y
  \]
  By \cref{parid}, this is zero.
\end{proof}

\begin{lemma}
  $\inner{a+b,c}=\inner{a,c}+\inner{b,c}$
  \label{lem:adding}
\end{lemma}
\begin{proof}
  Observe that
  \begin{align*}
    \inner{x+y,z}&=\normsq{x+y+z}-\normsq {x+y}-\normsq z\\
    \inner{x-y,z}&=\normsq{x-y+z}-\normsq {x-y}-\normsq z
  \end{align*}
  and consider the sum
  \[
  \inner{x+y,z}+\inner{x-y,z}=\normsq{x+z+y}+\normsq{x+z-y}-(\normsq{x+y}+\normsq{x-y})-2\normsq z
  \]
  By \cref{parid}, $\normsq{x+y}+\normsq{x-y}=2\normsq{x}+2\normsq{y}$ and $\normsq{x+y+z}+\normsq{x+y-z}=2\normsq{x+y}+2\normsq z$, so we get
  \[
  \inner{x+y,z}+\inner{x-y,z}=2\normsq{x+y}-2\normsq x - 2\normsq y
  \]
  which by \cref{parid} is
  \[
  \normsq{x+y}-\normsq{x-y}= \normsq{x+y}-2\normsq x-2\normsq y -(\normsq{x-y}-2\normsq{x}-2\normsq{y})
  \]
  By \eqref{parid} a final time, we get
  \[
  \inner{x+y,z}+\inner{x-y,z}=\inner{x,z}-\inner{-x,z}=2\inner{x,z}
  \]
  Now note that $\inner{2x,y}=\inner{x+x,y}-\inner{x-x,y}=2\inner{x,y}$ by the above. Thus 
  \[
  \inner{x+y,z}+\inner{x-y,z}=\inner{2x,y}
  \]
  Letting $x=\frac{a+b}{2}$, $y=\frac{a-b}2$, and $z=c$ proves the lemma.
\end{proof}

\begin{theorem}[Linearity]
  $\inner{kx+y,z}=k\inner{x,z}+\inner{y,z}$
\end{theorem}
\begin{proof}
  By \cref{lem:adding}, it suffices to show that $\inner{kx,y}=k\inner{x,y}$. To see this, first note it is trivial when $k=0$. Then consider the case where $k$ is a positive integer. We have 
  \[
  \inner{kx,y}=\inner{x+(k-1)x,y}=\inner{x,y}+\inner{(k-1)x,y}
  \]
  so it follows by induction that $\inner{kx,y}=k\inner{x,y}$. Now consider $\inner{k^{-1}x,y}$. By the previous result, we know $k\inner{k^{-1}x,y}=\inner{x,y}$, concluding $\inner{k^{-1}x,y}=k^{-1}\inner{x,y}$.

  As any rational number $q=ab^{-1}$ where $a,b\in\Z$, $b\neq 0$, we see that $\inner{qx,y}=\inner{ab^{-1}x,y}=ab^{-1}\inner{x,y}=q\inner{x,y}$. By continuity, we can extend this to the case where $r\in\R$ is arbitrary, giving $\inner{rx,y}=r\inner{x,y}$.
\end{proof}
\end{document}

