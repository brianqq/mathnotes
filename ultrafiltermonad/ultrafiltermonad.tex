\documentclass[10pt]{scrartcl}
\usepackage{amssymb, amsmath}
\usepackage[amsmath,thmmarks]{ntheorem}
\usepackage{thmtools}
%\usepackage{cfr-lm}
%\usepackage{lmodern}
\usepackage{float}
\usepackage{url}
\usepackage{microtype}
\usepackage{xfrac}
\usepackage{euler}
\usepackage{fontspec}
\setmainfont{Linux Libertine O}
\setmonofont{Inconsolatazi4}
\usepackage{tikz}
\usepackage{tikz-cd}

%\usepackage[style=alphabetic,backend=bibtex]{biblatex}


\makeatletter
\newcommand{\openbox}{\leavevmode
  \hbox to.77778em{%
  \hfil\vrule
  \vbox to.675em{\hrule width.6em\vfil\hrule}%
  \vrule\hfil}}
\gdef\proofSymbol{\openbox}
\newcommand{\proofname}{Proof.}
\newcounter{proof}\newcounter{currproofctr}\newcounter{endproofctr}%
\newenvironment{proof}[1][\proofname]{
  \th@nonumberplain
  \def\theorem@headerfont{\ttfamily}%
  \normalfont
  %\theoremsymbol{\ensuremath{_\blacksquare}}
  \@thm{proof}{proof}{#1}}%
  {\@endtheorem}
\makeatother

\setkomafont{disposition}{\sffamily\bfseries\color{gray}}


\usepackage{nameref}

\usepackage[colorlinks=true,linkcolor=blue,citecolor=blue,urlcolor=blue]{hyperref}
\renewcommand{\ref}{\hyperref}

\usepackage{cleveref}

\newcounter{thms}
\numberwithin{thms}{section}
\theoremstyle{definition}
\theoremheaderfont{\ttfamily}
\theorembodyfont{\normalfont}
\newtheorem{defn}[thms]{Definition}
\newtheorem{defcat}[thms]{Category definition}
\newtheorem{example}[thms]{Example}

\newtheorem{counterex}[thms]{Counter example}

\newtheorem{theorem}[thms]{Theorem}
\newtheorem{lemma}[thms]{Lemma}
\newtheorem{cor}[thms]{Corollary}

\crefname{lemma}{lemma}{lemmas}
\crefname{theorem}{theorem}{theorems}

\newcommand{\Z}{\mathbb{Z}}
\newcommand{\N}{\mathbb{N}}
\newcommand{\R}{\mathbb{R}}
\newcommand{\C}{\mathbb{C}}
\newcommand{\E}{\mathbb{E}}

\newcommand{\define}{:\equiv}
\DeclareMathOperator{\cl}{cl}

%categories
\newcommand{\cattop}{\mathtt{top}}
\newcommand{\ab}{\mathtt{Ab}}

\renewcommand{\lim}{\operatorname{lim}}
 

\AtBeginDocument{
  \pdfhorigin=\dimexpr\pdfhorigin-1.5in\relax
  \pdfvorigin=\dimexpr\pdfvorigin-1in\relax
  \pdfpagewidth=\dimexpr\pdfpagewidth-3in\relax
  \pdfpageheight=\dimexpr\pdfpageheight-3in\relax
}
\newcommand{\ultra}{\operatorname{\mathtt{ultra}}}
\begin{document}
\section{ultrafilter}
\begin{defn}[Filter]
  F is a filter iff
  \label{defn:filter}
  \begin{equation}
    \label{eq:filter-ax:finite-int}
    A,B\in F \implies A\cap B \in F
  \end{equation}
  \begin{equation}
    \label{eq:filter-ax:proper}
    \emptyset \notin F
  \end{equation}
  \begin{equation}
    \label{eq:filter-ax:upward-closed}
    B\supseteq A\in F \implies B\in F
  \end{equation}
\end{defn}

\begin{defn}[Ultrafilter]
  \begin{equation}
    \label{eq:ultra-ax}
    A\subseteq X \implies A \in U \textrm{ or } X-A\in U   
  \end{equation}
\end{defn}

\begin{theorem}
  Ultrafilter convergence defines a topology. 
\end{theorem}
\begin{theorem}
  A space is
  \begin{enumerate}
  \item compact iff every ultrafilter converges.
  \item Hausdorff iff every ultrafilter converges to at most one point.
  \item compact-Hausdorff iff every ultrafilter converges to exactly one
    point
  \end{enumerate}
\end{theorem}

\section{Stone-Čech compactification}
Let $X$ be a space. 
\begin{defn}[$\ultra$]

\end{defn}

\begin{defn}[$\ultra:A \to \ultra A$]
  Suppose $a\in A$. Let $\ultra a$ be the ultrafilter generated by $a$. Suppose $f:A\to B$ is a function. Then $\ultra (f):\ultra(A)\to\ultra(B)$. 
\end{defn}

\begin{defn}[$\lim: \ultra^2 X \to \ultra X$]
  Let $U\in \ultra^2 X$. Suppose $U=\{U_i\}$ where $U_i$ a subset of $\ultra X$. 
  Then define  
  \begin{equation}
    \lim U = \bigcup_{V\in U}\bigcap_{u\in V} u
    \label{eq:def:lim}
  \end{equation}
\end{defn}

\begin{theorem}
  $\lim U:\ultra^2 X \to \ultra X$
\end{theorem}

\begin{proof}
  I claim 
  \begin{equation}
    \label{eq:def:Fi}
    F_i\define\bigcap_{u \in U_i} u
  \end{equation}
  is a filter. To see this, note every filter on $X$ contains $X$. Hence $F_i$ is nonempty. The other filter axioms remain true after taking intersections. 
  
  Now pick arbitrary $F_\alpha$ and $F_\beta$ as defined in \cref{eq:def:Fi}. I claim there is some $F_\gamma$ (and corresponding $U_\gamma$) such that $F_\alpha \cup F_\beta\subseteq F_\gamma$. It suffices to find the corresponding $U_\gamma$. Let $U_\gamma=U_\alpha\cap U_\beta$. \Cref{eq:filter-ax:finite-int} guarantees $U_y\in U$. Then 
  \begin{align*}
  \bigcap_{u\in U_\gamma} u &\supseteq \left(\bigcap_{u\in U_\alpha} u\right) \cup \left(\bigcap_{u\in U_\beta}u\right)\\
    F_\gamma &\supseteq F_\alpha \cup F_\beta
  \end{align*}
  Hence
  \[
  \lim U \define\bigcup_{i} F_i 
  \]
  is a filter. 
  
  Finally, I claim $\lim U\in \ultra X$. Suppose $A\subseteq X$. Then let $\ultra A\subseteq \ultra X$ be the set of ultrafilters that contain $A$. I claim $\ultra (A^C) =(\ultra A)^C$; as $(\ultra A)^C$, the set of ultrafilters that \emph{do not contain} $A$, is $\ultra(A^C)$, the set of ultrafilters that  
\emph{contain} $A^C$. This is a rewording of \cref{eq:ultra-ax}.

By \cref{eq:ultra-ax}, either $\ultra A\in U$ or $\ultra (A^C)\in U$. Suppose, without loss of generality, that $\ultra A\in U$. Then
\[
F_A:=\bigcap _{u \in \ultra A} u \subseteq \bigcap_ {u\in U} u 
\]
Note $F_A$ is a filter containing $A$. Then $\lim U$ contains $A$. \Cref{eq:ultra-ax} holds. 
\end{proof}

\begin{theorem}[$\lim \circ \ultra = id$]
  Let $u\in \ultra X$. Then
  $\lim\circ  \ultra u = u$ as every set containing $u$ is in $\ultra u$. 
\end{theorem}

\section{ultrafilter monad}
For this section, we will distinguish between $\ultra$ and $\eta$. Here, $\ultra:\texttt{haus}\to \texttt{Chaus}$ is the monad's functor and (via abuse of notation) $\ultra:id\to \ultra$ is the unit and $\lim:\ultra^2\to \ultra$ is the multiplication. 
\begin{theorem}
  The coherence conditions
  \begin{equation}
    \label{cohere-assoc}
    \begin{tikzcd}
      \ultra^3(X)\rar{\ultra(\lim_X)}\dar{\lim_{\ultra(X)}} & \ultra^2(X)\dar{\lim_X}  \\
      \ultra^2(X)\rar{\lim_X}& \ultra(X)
    \end{tikzcd}
  \end{equation}
  and 
  \begin{equation}
    \label{cohere-id}
    \begin{tikzcd}
      \ultra(X)\rar{\ultra_{\ultra(X)}}\dar{\ultra(\ultra_X)}& \ultra^2(X)\dar{\lim_X}\\
      \ultra^2(X)\rar{\lim_X}&\ultra(X)
    \end{tikzcd}
\end{equation}
hold
\end{theorem}
\begin{proof}
  Consider \cref{cohere-assoc}. Let $\mathcal{U}\in \ultra^3 X$. The goal is to show \[\lim\circ \ultra(\lim) \mathcal U = \lim\circ\lim \mathcal U\]
  First, consider $\lim \circ \ultra(\lim)\mathcal U$. 
  Then

  \begin{align*}
  \lim \circ \ultra(\lim) &= \bigcup_{U\in\mathcal U}\left(\bigcap_{u\in \bigcup_{V\in U} \bigcap_{v\in V} v}u \right) \\
                          &= \bigcup\bigcap\left(\bigcup_{\bigcap}\right) \\
                          &= \bigcup_{U\in\mathcal U}\bigcap_{V\in U} \bigcup_{u\in V}\bigcap_{v\in u} v \\
    \lim\circ\lim &= \bigcup_{U\in\mathcal U}\bigcap_{V\in U} \bigcup_{u\in V}\bigcap_{v\in u} v
  \end{align*}
  
  Now consider \cref{cohere-id}. This follows as $\ultra \textrm{ (unit) } = \ultra \textrm{ (functor) }$. 
\end{proof}

\section{Maximality}
Consider the space $X$ and its compactification $\bar X$. We have the diagram
\[
\begin{tikzcd}
  X\rar[hook]{i}\dar{ultra} & \bar X\dar[equal] \\
  \ultra{X} \rar{\ultra(i)} & \ultra{\bar{X}}
\end{tikzcd}
\]
I claim surjectivity of 
\[
\lim_{\bar X}\circ\ultra(i):\ultra X\to \bar X
\]
As $\bar X$ is compact, $\lim_{\bar X}$ is a homeomorphism. Thus it suffices to show the surjectivity of $\ultra(i)$. To see this, consider an ultrafilter $U\in\bar{X}$. As $\bar X$ is a compactification, $U$ is uniquely determined by $U|_X$. Thus $\ultra(i)\left(U|_x\right)=U$.
\end{document}
%%% Local Variables:
%%% mode: latex
%%% TeX-master: t
%%% End:
